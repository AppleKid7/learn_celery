\documentclass[9pt]{beamer}
\usepackage{minted}
%\usemintedstyle{manni}
\usemintedstyle{murphy}
\usepackage{hyperref}
\hypersetup{
colorlinks=true,
urlcolor=blue
}
\usepackage{graphicx}

\begin{document}
\title{How to use Celery+RabbitMQ}
\author{Nick Thompson} 
\date{\today}

\frame{\titlepage}

\begin{frame}[fragile]
\frametitle{Getting started:}
\begin{minted}{bash}
$ git clone https://github.com/NAThompson/learn_celery.git
$ cd learn_celery
$ pip3 install -r requirements.txt
\end{minted}
\end{frame}

\begin{frame}[fragile]
  \frametitle{What will we attempt to accomplish?}
  
  \begin{itemize}
  \item Link Celery+RabbitMQ into Django ecosystem
    \pause
  \item Use tasks to keep page loads from timing out
    \pause
  \item Associate a running job with a progress bar
    \pause
  \item Link CPU cycles to billing
    \pause
  \item Autoscale nodes based on queue length
  \end{itemize}
\end{frame}

\begin{frame}
  \frametitle{Why Celery + RabbitMQ}
  \begin{itemize}
  \item Job scheduling is boring, and it's a pain.
    \pause
  \item So you certainly don't want to learn \emph{two} job schedulers
    \pause
  \item You never want to be in a spot where you have to swap out your job scheduler
    \pause
  \item Therefore, use the most well-tested and popular stack
  \end{itemize}
\end{frame}

\begin{frame}
  \frametitle{What is RabbitMQ?}
  \begin{itemize}
  \item RabbitMQ is an implementation of the \emph{advanced messaged queuing protocol} (AMQP), a standardized way to pass data between applications.
  \item The specification of AMQP is \href{http://www.amqp.org/specification/0-10/amqp-org-download}{300 pages long}, but it's basically an email server protocal for binary applications
  \end{itemize}
\end{frame}

\begin{frame}[fragile]
  \frametitle{How do I set up RabbitMQ?}
  \begin{minted}{bash}
    $ sudo apt-get install -y rabbitmq-server
    $ sudo rabbitmqctl status
    $ sudo lsof -i :5672 # 5672 is the well-known port of AMQP
  \end{minted}
\end{frame}

\begin{frame}
  \frametitle{Testing RabbitMQ}
  We want to get the minimal working RabbitMQ working on localhost:
\end{frame}

\end{document}
